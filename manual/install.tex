<<<<<<< .mine

\chapter{Installation}
The \libalf library contains all functionality to embed different kinds of (offline and/or online algorithms) into your learning application. It can be used as \cpp library directly, via JNI (Java Native Interface) or via a dedicated client/server application called the dispatcher. Both may be used from Java interchangable via the generic interface ``jalf''.
The chapter gives information on how to compile \libalf and shows how to set up \libalf in your system to use in a user application under Windows and Linux operating systems.

\section{Package Information}
\begin{enumerate}
 \item \textbf{src}
	\libalf source files. 
 \item \textbf{include}
	Include files for all learning algorithms, knowledgebase, loggers and normalizers.
 \item \textbf{jalf}
	The java implementation consisting of source files, includes and testsuites. 
 \item \textbf{lib}
 	
 \item \textbf{dispatcher}
 	Necessary files for implementing dispatcher consisting of the source files, indcludes and testsuites.
 \item \textbf{testsuites}
        The testsuites for all learning algorithms implemented in \libalf.
\end{enumerate}

\section{Requirements}
\libalf is a platform independent, stand-alone \cpp library and does not depend on any other libraries except the \cpp standard library and standard template library (STL). Also, MiniSat v1.14 is fully integrated into libalf, as it is required for the bierman learning algorithm. There are no further dependencies. It has been tested successfully in x86 and x64 systems, under Linux, Windows and Mac Operating Systems. To use \libalf, your computer must have the following softwares installed.
\begin{enumerate}
 \item \textbf{GNU Compiler} \vskip 1pt
	For systems running Linux, the gcc compiler available by default is sufficient. \vskip 1pt
	For Windows, we recommend the use of MinGW compiler. Instructions on where to find it and how to install it are given below.
 \item \textbf{Java 1.6 (or later version)} \vskip 1pt
	This is required to use \libalf via JNI. \jalf can be compiled using \texttt{ant}.
\end{enumerate}
\paragraph{Note:}
The dispatcher currently compiles only under POSIX compatible systems. 

\section*{Installing MinGW compiler in Windows}
We recommend the usage of MinGW compiler for users who would like to use \libalf in Windows. 

\subsection*{Where to find it and how to install it}
The downloadable package and installation instructions of the MinGW compiler can be found in the following website.
\url{http://www.mingw.org/} \vskip 1pt

\textbf{Note:} Installing MinGW compiler on your computer does not automatically set the ``PATH'' variable. You may have to manually update the ``PATH'' variable of the computer to the ``bin'' folder of MingW. (Usually ``C:\textbackslash MingW\textbackslash bin'', unless you have changed the directory of installation). 

After installing the MinGW compiler, ensure that the following modifications are made to be able to compile \libalf.

\subsection*{Modification to the MinGW compiler}

As a prerequisite towards compiling the library a small change has to be made to the ``bin'' folder of the MingW.   
After installing MinGW compiler, three files under the ``/bin'' folder are to be renamed as follows.
\begin{enumerate}
 \item \textbf{mingw32-make.exe} to \textbf{make.exe}
 \item \textbf{mingw32-gcc.exe} to \textbf{gcc.exe}
 \item \textbf{mingw32-ar.exe} to \textbf{ar.exe}
 \item \textbf{mingw32-ld.exe} to \textbf{ld.exe}	
\end{enumerate}
 
\section{The \libalf \cpp Library}
The section describes how to compile the sources and use the \cpp library directly in your application.
\subsection*{Compiling \libalf}
A ``Makefile'' is already provided to compile \libalf.
\paragraph{}
To compile the library in linux, use the following command.
\marginpar{\hspace{55pt} \textbf{Linux}}
\[
  PREFIX = <path> make
\]
\emph{$<$path$>$} must specifiy the directory where the \libalf sources are located. This is because \texttt{make} searches the \emph{include} directory inside the directory specified by PREFIX for the \libalf's header files. If you do not wish to use gcc compiler or if gcc is not installed in /usr/bin/gcc, you can additionally specify the CC variable pointing to your desired \cpp compiler. \vskip 1pt
\reversemarginpar
\marginpar{\hspace{35pt} \textbf{Windows}} To compile the library in Windows, enter the following command in \textbf{Command Prompt}.

\[
 make -f <path>/makefile
\]
\subsection*{Employing \libalf in an Application}
To use the installed library in your application, follow the steps given below. \vskip 1pt

On Linux, you use the LD\_LIBRARY\_PATH to point to \libalf's location.
\marginpar{\hspace{55pt} \textbf{Linux}}
\[
  LD\_LIBRARY\_PATH = <path>/bin
\]

On windows, one of the following methods can be used 
\marginpar{\hspace{35pt} \textbf{Windows}} 
\begin{enumerate}
 \item Place the compiled library into the directory where your application is located.
 \item Alternatively, you can add the path containing the \libalf library in to your \textbf{path} variable.
\end{enumerate}

\section{The \jalf Java Library}
The section describes how to install the Java interface for \libalf and use them in your application.
\subsection*{Compiling \jalf}
To compile the Java based \jalf library, you may use ANT (\url{http://ant.apache.org/}). \vskip 1pt
In both Windows and Linux, the following command can be used.
\[
  ant -D PREFIX=<path>
\]
Here, the variable ``PREFIX'' specifies the directory where the ``jalf.jar'' is located. The option -D is used to specify variables (in this case it adds the ``jalf.jar'' to Java's classpath). \vskip 1pt

\subsection*{Employing \jalf in an Application}
To use \libalf in your application, you must make sure that ``jalf.jar'' can be found. You must also ensure that the compiled \jalf library containing the \libalf sources is also found.  \vskip 1pt
In both Windows and Linux, you can do this by specifying the following two parameters.
\begin{enumerate}
 \item -classpath ``$<$path$>$:.'' \vskip 1pt
	Use the command to set the classpath to point to ``jalf.jar''. \vskip 1pt
	\textbf{Note:} The seperator ``:'' must be used on Linux and ``;'' must be used on Windows.
 \item -D java.library.path = $<$path$>$ \vskip 1pt
	Use the command to let the java library path variable point to the compiled jalf library. 
\end{enumerate}

\section{Troubleshooting}
If you have problems compiling the library using \texttt{make} in windows, it could be because of Linux commands used in the makefiles. Ordinarily, MinGW should not have problems compiling it. But if problems do exist, then download and install \textbf{makeutils} which can be found at \url{http://makeutil.sourceforge.net/#download}. \vskip 1pt
Please make sure that the ``/bin'' folder of \textbf{makeutil} is added to your \texttt{path} variable. 

\section{License Information}
\libalf is a free software published under the LGPL v3 license.Under this license, you have the complete freedom for the following.
\begin{enumerate}
 \item Use the software for any purpose.
 \item Make copies and redistribute the software.
\end{enumerate}
You also gain the complete freedom to modify the software according to your needs and distribute the modified version of the software. However, the modified software must be published with the code under the LGPL v3 (or later version) license. 
For more information about LGPL license, visit \url{http://www.gnu.org/licenses/lgpl.html}



=======

\chapter{Installation}
The \libalf library contains all functionality to embed different kinds of (offline and/or online algorithms) into your learning application. It can be used as \cpp library directly, via JNI (Java Native Interface) or via a dedicated client/server application called the dispatcher. Both may be used from Java interchangable via the generic interface ``jalf''.
The chapter gives information on how to compile \libalf and shows how to set up \libalf in your system to use in a user application under Windows and Linux operating systems.

\section{Package Information}
\begin{enumerate}
 \item \textbf{libalf}
	The libalf learning framework
 \item \textbf{liblangen}
	A \cpp library for generating (regular) languages by means of randomly drawn finite automata or regular expressions. More explicitly, this library can currently generate random DFA, NFA and regular expressions.
 \item \textbf{finite automata tool}
	A command-line tool for creating, transforming and comparing automata.
 \item \textbf{libAMoRE(++)}
 	libAMoRE is a comprehensive automata library written in C. libAMoRE++ is a \cpp interface to libAMoRE, implementing several extensions and additional functionality.
 \item \textbf{libalf Demo}
 	A GUI application using libalf for employing various learning algorithms via dispatcher or JNI.
\end{enumerate}

\section{Requirements}
\libalf is a platform independent, stand-alone \cpp library and does not depend on any other libraries except the \cpp standard library and standard template library (STL). Also, MiniSat v1.14 is fully integrated into libalf, as it is required for the bierman learning algorithm. There are no further dependencies. It has been tested successfully in x86 and x64 systems, under Linux, Windows and Mac Operating Systems. To use \libalf, your computer must have the following softwares installed.
\begin{enumerate}
 \item \textbf{GNU Compiler} \vskip 1pt
	For systems running Linux, the gcc compiler available by default is sufficient. \vskip 1pt
	For Windows, we recommend usage of MinGW compiler for compiling \libalf. Instructions on where to find it and how to install it are given in section --.
 \item \textbf{Java 1.6 (or later version)} \vskip 1pt
	This is required to use \libalf via JNI. \jalf can be compiled using \texttt{ant}.
\end{enumerate}
\paragraph{Note:}
The dispatcher currently compiles only under POSIX compatible systems. 

\section*{Installing MinGW compiler in Windows}
We recommend the usage of MinGW compiler for users who would like to use \libalf in Windows. 

\subsection*{Where to find it and how to install it}
The downloadable package and installation instructions of the MinGW compiler can be found in the following website.
\url{http://www.mingw.org/} \vskip 1pt

\textbf{Note:} Installing MinGW compiler on your computer does not automatically set the ``PATH'' variable. You may have to manually update the ``PATH'' variable of the computer to the ``bin'' folder of MingW. (Usually ``C:\textbackslash MingW\textbackslash bin'', unless you have changed the directory of installation). 

After installing the MinGW compiler, ensure that the following modifications are made to be able to compile \libalf.

\subsection*{Modification to the MinGW compiler}

As a prerequisite towards compiling the library a small change has to be made to the ``bin'' folder of the MingW.   
After installing MinGW compiler, three files under the ``/bin'' folder are to be renamed as follows.
\begin{enumerate}
 \item \textbf{mingw32-make.exe} to \textbf{make.exe}
 \item \textbf{mingw32-gcc.exe} to \textbf{gcc.exe}
 \item \textbf{mingw32-ar.ext} to \textbf{ar.exe}	
\end{enumerate}
 
\section{The \libalf \cpp Library}
The section describes how to compile the sources and use the \cpp library directly in your application.
\subsection*{Compiling \libalf}
A ``makefile'' is provided already to compile \libalf.
\paragraph{}
To compile the library in linux, use the following command.
\marginpar{\hspace{55pt} \textbf{Linux}}
\[
  PREFIX = <path> make
\]
\emph{$<$path$>$} must specifiy the directory where the \libalf sources are located. This is because \texttt{make} searches the \emph{include} directory inside the directory specified by PREFIX for the \libalf's header files. If you do not wish to use gcc compiler or if gcc is not installed in /usr/bin/gcc, you can additionally specify the CC variable pointing to your desired \cpp compiler. \vskip 1pt
\reversemarginpar
\marginpar{\hspace{35pt} \textbf{Windows}} To compile the library in Windows, enter the following command in \textbf{Command Prompt}.

\[
 make -f <path>/makefile
\]
\subsection*{Employing \libalf in an Application}
To use the installed library in your application, follow the steps given below. \vskip 1pt

On Linux, you use the LD\_LIBRARY\_PATH to point to \libalf's location.
\marginpar{\hspace{55pt} \textbf{Linux}}
\[
  LD\_LIBRARY\_PATH = <path>/bin
\]

On windows, one of the following methods can be used 
\marginpar{\hspace{35pt} \textbf{Windows}} 
\begin{enumerate}
 \item Place the compiled library into the directory where your application is located.
 \item Alternatively, you can add the path containing the \libalf library in to your \textbf{path} variable.
\end{enumerate}

\section{The \jalf Java Library}
The section describes how to install the Java interface for \libalf and use them in your application.
\subsection*{Compiling \libalf}
To compile the Java based \libalf library, you may use ANT (\url{http://ant.apache.org/}). \vskip 1pt
In both Windows and Linux, the following command can be used.
\[
  ant -D PREFIX=<path>
\]
Here, the variable ``PREFIX'' specifies the directory where the ``jalf.jar'' is located. The option -D is used to specify variables (in this case it adds the ``jalf.jar'' to Java's classpath). \vskip 1pt

\subsection*{Employing \jalf in an Application}
To use \libalf in your application, you must make sure that ``jalf.jar'' can be found. Secondly, you must also ensure that the compiled \jalf library containing the \libalf sources is also found.  \vskip 1pt
In both Windows and Linux, you can do this by specifying the following two parameters.
\begin{enumerate}
 \item -classpath ``$<$path$>$:.'' \vskip 1pt
	Use the command to set the classpath to point to ``jalf.jar''. \vskip 1pt
	\textbf{Note:} The seperator ``:'' must be used on Linux and ``;'' must be used on Windows.
 \item -D java.library.path = $<$path$>$ \vskip 1pt
	Use the command to let the java library path variable point to the compiled jalf library. 
\end{enumerate}

\section{Troubleshooting}
If you have problems compiling the library using \texttt{make} in windows, it could be because of Linux commands used in the makefiles. Ordinarily, MinGW should not have problems compiling it. But if problems do exist, then download and install \textbf{makeutils} which can be found at \url{http://makeutil.sourceforge.net/#download}. \vskip 1pt
Please make sure that the ``/bin'' folder of \textbf{makeutil} is added to your \texttt{path} variable. 

\section{License Information}
\libalf is a free software published under the LGPL v3 license.Under this license, you have the complete freedom for the following.
\begin{enumerate}
 \item Use the software for any purpose.
 \item Make copies and redistribute the software.
\end{enumerate}
You also gain the complete freedom to modify the software according to your needs and distribute the modified version of the software. However, the modified software must be published with the code under the LGPL v3 (or later version) license. 
For more information about LGPL license, visit \url{http://www.gnu.org/licenses/lgpl.html}



>>>>>>> .r1127
