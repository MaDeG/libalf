\chapter{Loggers \& Statistics}
\section{logger}

To ease application development and debugging, \libalf provides two components - Loggers and Statistics. 

A \textbf{Logger} is an adjustable logging facility that an algorithm can write to. One may insert different category of messages in the logger and is of substantial use in application debugging. When a learning algorithm is initialized, a logger is associated with it along with the knowledgebase. \libalf provides flexible logger implementations for the user. A learning algorithm can either use an output stream or a buffer as the logger. On the contrary, one may also choose to work without a logger.

\textbf{Statistics} refers to the statistical data that can be acquired by evaluating the learning procedure. Information about the memory usage, queries produced, time taken for computing conjecture and other details may serve as base for analysing the learning algorithm in various cases. 

\section{Loggers}

A learning algorithm may write different types of messages to the logger. A logger has to be associated with the knowledgebase during the initialization of the learning algorithm. 

\subsection*{Example of how to set a logger}
The following code snippet shows to how to create an instance of a logger and associate it with the learning algorithm along with knowledgebase and alphabet size.
\begin{lstlisting}
// getting the alphabet size
int alphabet_size = get_AlphabetSize(); 

// create instance of knowledgebase
knowledgebase<bool> base; 

// create instance of a buffered logger.
buffered_logger bufflog; 

// Create learning algorithm (Angluin L*) 
// with a logger - bufflog and alphabet size - alphabet_size
angluin_simple_table<bool> algorithm(&base, bufflog, alphabet_size); 
\end{lstlisting}

\subsection{The concept of Loglevel}

To achieve a good organization of these messages written to the logger, the messages are categorized with different labels. A  \texttt{loglevel} is a marker indicating the priority of types of messages to be written into the logger. Setting up a \texttt{loglevel} ensures only messages having priority of that level and higher are written to the logger while messages having lower priority are discarded or skipped.
\paragraph{}
The category of messages is initialized with a \emph{enum} type variable \texttt{logger\_loglevel}.
\begin{itemize}
 \item \textbf{LOGGER\_INTERNAL=0} ; (An internal method)
 \item \textbf{LOGGER\_ERROR = 1} ; All log messages that describe a non-recoverable error are marked with this.
 \item \textbf{LOGGER\_WARN = 2} ; Messages describing a state or command that is erroneous but may be ignored under most conditions.
 \item \textbf{LOGGER\_INFO = 3} ; Any information that does not describe an erroneous condition.
 \item \textbf{LOGGER\_DEBUG = 4} ; Messages that may help debugging of libalf.( Most likely removed before release version ).
 \item \textbf{LOGGER\_ALGORITHM = 5} ; (Do not use this as minimal loglevel)
\end{itemize}
For instance, setting up a \texttt{loglevel} of ``2'' will make the learning algorithm write warning and error messages (level 1 and 2) to the logger while messages labelled with lower priority (3 and 4) and discarded.

\subsection{The Logger class}
The main class that consists of attributes and methods to implement the logger. 

\subsection*{Attributes} 
It consists of two attributes that every type of logger makes use of.
\begin{enumerate}
 \item \textbf{enum logger\_loglevel minimal\_loglevel} - a minimal setting of the loglevel.
 \item \textbf{bool log\_algorithm} - An boolean variable to indicate if a logger is to be associated with an algorithm or not.
\end{enumerate}

\subsection*{Methods}
The following methods are defined in the class.
\begin{enumerate}
 \item \textbf{void set\_minimal\_loglevel(enum logger\_loglevel minimal\_loglevel)} \\
	Sets the minimum logger level using the \texttt{loglevel} attributes.
 \item \textbf{void set\_log\_algorithm(bool log\_algorithm)} \\
	The method sets logger for the algorithm if the argument is \true. It ignores the logger if the parameter is \false. 
 \item \textbf{virtual void operator()(enum logger\_loglevel, string\&)} and \textbf{virtual void operator()(enum logger\_loglevel, 	const char* format, ...)} \\
	The method takes the logger type and message as parameters for entry to the logger. If other variables also need to be used, it can be done so using the second method. 
\end{enumerate}

\subsection*{Example of a learning algorithm writing message to the logger - Developer's View}
\begin{lstlisting}
if(my_knowledge == NULL) 
{
(*my_logger)(LOGGER_ERROR, ``learning_algorithm::advance() 
     no knowledgebase was set!\n'');
return false; 
}
\end{lstlisting}
The above code snippet is an extraction from the ``advance()'' method of a learning algorithm. When no knowledgebase is set to the algorithm, it enters the message ``learning\_algorithm::advance(): no knowledgebase was set'' to the log ``my\_logger'' and marks it as an error with ``LOGGER\_ERROR''. 
\vskip 1pt
All three types of loggers are implemented with the respective classes, \texttt{ignore\_logger}, \texttt{ostream\_logger} and \texttt{buffered\_logger}. All the classes inherit the \texttt{logger} class. \\

\subsection{Types of Loggers}
\label{sec:types}

\subsection*{ignore\_logger}
A class that does not consist of any methods. In this case, the logger simply discards all messages.

\subsection*{ostream\_logger}
The class consists of method to write the message to an output stream. 
\begin{itemize}
 \item \textbf{ostream\_logger::ostream\_logger(ostream *out, enum logger\_loglevel minimal\_loglevel, bool log\_algorithm = true, bool use\_color = true)} \vskip 1pt
	The method creates an output stream for the logger. The parameter \texttt{*out} points to the output stream and the minimal loglevel indicates the initial setting. The parameter \texttt{log\_algorithm} is set to \true by default since the logger will be used by the algorithm. The parameter \texttt{use\_color} is set to \true so that on a console output, you may view the messages in different colors!
\end{itemize}

\subsection*{buffered\_logger}
The class consists of methods for setting a buffer as a log (typically a string). It should be noted that the messages passed to the buffer will not be available until it is received and flushed explicitly.

\begin{enumerate}
 \item \textbf{buffered\_logger::buffered\_logger(enum logger\_loglevel minimal\_loglevel, bool log\_algorithm = true)} \\
	The method sets the buffered logger with the minimal loglevel. \texttt{log\_algorithm} is set to \true.
 \item \textbf{buffered\_logger::string * receive\_and\_flush()} \\
	The method receives and flushes the buffered stream.
\end{enumerate}

\section{Statistics}
The component is used to obtain three types of statistical information. They are organized as three classes namely 
\begin{itemize}
\item \texttt{timing\_statistics} - To retrieve the user time and system time of the learning process.
\item \texttt{memory\_statistics} - To retrieve information about the memory usage
\item \texttt{query\_statistics} - To retrieve information about the queries produced during the learning procedure.
\end{itemize}

For example, memory statistics may be helpful in deciding whether a clean up is required for the knowledgebase. Timing statistics and query statistics help to analyse the performance of the learning algorithm. In addition to their uses, they support serialization and deserialization as another advantage. 

\paragraph{}
These classes are defined in the following manner.

\subsection{class timing\_statistics}
The class consists of methods to obtain timing statistics of the learning procedure.
\subsection*{Attributes}
\begin{enumerate}
 \item \textbf{int32\_t user\_sec} - Variable to store the User time.
 \item \textbf{int32\_t user\_usec}
 \item \textbf{int32\_t sys\_sec} - Variable to store the System time.
 \item \textbf{int32\_t sys\_usec}
\end{enumerate}
\subsection*{Methods}
\begin{enumerate}
 \item \textbf{timing\_statistics()} \\
	Constructor that retrieves the timing information.
 \item \textbf{void reset()} \\
	The method resets all the acquired timing statistics.
\end{enumerate}

\subsection{class memory\_statistics}
The class consists of methods to obtain statistical information about the memory usage during the learning procedure. 
\subsection*{Attributes}
Most of the attributes belonging to this class are used by the Angluin algorithm.
\begin{enumerate}
 \item \textbf{int32\_t bytes} - bytes of algorithms data structure
 \item \textbf{int32\_t members} - number of membership data
 \item \textbf{int32\_t words} - number of words in table
 \item \textbf{int32\_t upper\_table} - size of upper table (if appropriate)
 \item \textbf{int32\_t lower\_table} - size of lower table (if appropriate)
 \item \textbf{int32\_t columns} - columns (if appropriate)
\end{enumerate}
\subsection*{Methods}
\begin{enumerate}
 \item \textbf{memory\_statistics()} \\
	Constructor to retrieve the memory statistics.
 \item \textbf{void reset()} \\
        The method resets all the acquired memory statistics.
\end{enumerate}

\subsection{class query\_statistics}
The class consists of methods to obtain statistical information about the queries produced during the learning procedure.
\subsection*{Attributes}
\begin{enumerate}
 \item \textbf{int32\_t membership} - Stores the number of membership queries.
 \item \textbf{int32\_t uniq\_membership}
 \item \textbf{int32\_t equivalence} - Stores the number of equivalence queries.
\end{enumerate}
\subsection*{Methods}
\begin{enumerate}
 \item \textbf{query\_statistics()} \\
	Constructor that retrieves statistics about the all queries produced.
 \item \textbf{void reset()}
	The method resets all the acquired statics about queries.
\end{enumerate}

\subsection{class statistics}
The main class that contains objects of the other three classes (the three types of statistics) to access the methods. 
\subsection*{Attributes}
\begin{enumerate}
 \item \textbf{query\_statistics queries} - Object to retrieve query statistics.
 \item \textbf{memory\_statistics memory} - Object to retrieve memory statistics.
 \item \textbf{timing\_statistics time} - Object to retrieve the timing statistics.
\end{enumerate}
\subsection*{Methods}
Along with the serialize and deserialize, the class defines the following method.
\begin{itemize}
 \item \textbf{void reset()}
	The method resets all the statistical information.
\end{itemize}
