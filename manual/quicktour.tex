\chapter{Quick Tour of Libalf}
This document is a quick guide for understanding \libalf in the context of how to use it in your application. In the following sections we present a brief description of the library and important features.

\section{Introduction}
The central working component of \libalf is the learning algorithm (which essentially is a set of learning algorithms implemented in the library listed in table \ref{listofalgs}) and its main repository is the knowledgebase which contains all knowledge required for the learning process. \\
In brief, the learning algorithm does an ``advance'' and examines the words and their classifications available in the knowledgebase and tries to build a conjecture from it. \\

\subsection{The Knowledgebase}
The knowledgebase stores the words and their classification. \libalf uses 32 bit integers to store the words. Internally. these words are stored as \texttt{list of integers}. \\
Classification refers to a value that is mapped to every word. For instance, the classification of a word for a Finite Automata may either be \true or \false which indicates whether the word is accepted in the language or not. \\
Since the repository is a template class, you can use arbitrary values for storing the classification. But the \jalf java library and dispatcher only allows boolean values for storing the classification. 
\paragraph{Conjecture} //This section is not so clear \\
The above stated information form the basic knowledge for the learning algorithms to compute a conjecture. A conjecture is a collection of states and transitions. The simple\_automaton class defines the data structures for forming the conjecture. The algorithm creates a \emph{simple automaton} with preliminary states when it is initialized. The final conjecture is built by adding \emph{state transitions}. That is, a word classified as \true will reach the end state through the state transitions. 

\section{The Learning Algorithms}
Learning algorithms communicate with the knowledgebase for getting information about words and their classification. They are classified as \online and \offline follow slightly differing techniques in building the conjecture. The algorithms perform an \emph{advance}, wherein it takes information from the knowledgebase and tries to construct a conjecture out of it. The offline algorithm, at this stage, either builds a complete conjecture or fails. The difference between the \online and \offline algorithms comes from this aspect. While an \offline algorithm performs an \emph{advance} only once, the \online algorithm would typically perform that more than once. Given below is the list of learning algorithms implemented in \libalf.

\begin{table} [h]
\label{listofalgs}
\centering
\begin{tabular}[c]{lcr}
\toprule[1pt]
Online Algorithms & Offline Algorithms \\	
\midrule
Angluin's L [2] (two variants) & Biermann [3] \\
NL [4] & RPNI [13] \\
Kearns / Vazirani [10] & DeLeTe2 [6]\\
\bottomrule[1pt]
\end{tabular}
\caption{List of Algorithms Implemented}
\label{algtables1}
\end{table}

\subsection{Online learning algorithm}
The \online learning algorithm computes the conjecture by iteration of advance operation followed by queries asked to the teacher, where teacher refers to the user application. Queries are either membership queries or equivalence queries. Membership queries refer to unknown classification of a word that needs to be \emph{resolved}, i.e.\ the classification has to be provided. Equivalence query refers to correctness of a computed conjecture asked to the teacher and when it is incorrect, a counter-example is provided to the algorithm and the iteration continues. The workflow of such an algorithm is as follows. \\

\begin{enumerate}
 \item The algorithm is furnished with the pointer to the knowledgebase and the alphabet size.
 \item The following two steps are repeated until a correct conjecture is determined.
	\begin{enumerate}
	    \item The algorithm is made to advance. In this step, the algorithm checks if there is enough information available in the knowledgebase to formulate a conjecture.
	    \item Here one of the following two possible events may occur.
	    \begin{enumerate}
	    \item If no conjecture was created, typically membership queries are produced and added to the knowledgebase. These queries are then presented to the teacher (user application) to resolve it.
	    \item On the other hand, if a conjecture was created, the equivalence query is answered by the teacher. If the conjecture was incorrect a counter example is rendered by the teacher.
	    \end{enumerate}
	\end{enumerate}
\end{enumerate}
\paragraph{}
In the online technique, the algorithm produces queries from an empty string to possible words out of the alphabet size until a conjecture can be evaluated. This implies that examples are provided to the learning algorithm on-the-fly. This is the major difference  from the offline technique.

\subsection{Example Code}
Below is a simplified example code to outline how to work with an \online algorithm. Refer our web page for more detailed information on the same. 
\lstset{language=c++, numbers=left, numberstyle=\tiny, stepnumber=1, numbersep=5pt}
\begin{lstlisting}[frame=single]
void main(int argc, char**argv) {

//Input the alphabet size
int alphabetsize = get_AlphabetSize();

//Create a knowledgebase. Observe that we choose 'bool' 
as its type.
knowledgebase<bool> base;

//Creating a learning algorithm with the knowledgebase, a
 buffered logger and the obtained alphabet size.
angluin_simple_table<bool> algorithm(&base,
buffered_logger bufflog,alphabetsize);

// The iteration of 'advance' followed by resolving of queries
do {
 conjecture * cj = algorithm.advance();
 if (cj == NULL) 
   // resolve membership queries
 else 
   // resolve equivalence query and give counter example.
}while (result == NULL); //where the variable "result" stores 
                         //the conjecture if it is correct.
}
\end{lstlisting}
% why should the alphabet size be given first in offline?

\subsection{Offline learning algorithm}
On the contrary to the \online technique, the \offline algorithm computes the conjecture by performing an advance only once from an already provided set of examples, which is a list of words and their classification, available in the knowledgebase. \\
Note: While an \offline algorithm guarantees that the automaton is consistent with the words, it makes no statement about how unknown words are classified. 
\paragraph{}
The online algorithm works as follows.
\begin{enumerate}
 \item The alphabet size is provided.
 \item The knowledgebase is furnished with a set of examples.
 \item The learning algorithm is made to advance to compute the conjecture in conformance with the examples.
\end{enumerate}
\paragraph{}

\subsubsection*{Example Code}
For a change, we show you a java example below that uses the RPNI \offline algorithm. The complete example can be found in our web page.
\lstset{language=java, numbers=left, numberstyle=\tiny, stepnumber=1, numbersep=5pt}
\begin{lstlisting}[frame=single]
public static void main(String[] args) throws 
NumberFormatException,IOException 
{
// A LibALFFactory called "factory" is created and 
set to be STATIC.
 LibALFFactory factory = JNIFactory.STATIC;

// A Knowledgebase under the name of "base" is created in the factory.
Knowledgebase base = factory.createKnowledgebase();

// Information about the size of the alphabet is obtained 
from the user.
alphabetsize = get_AlphabetSize();

while (input.equals("y")) {
// Samples collected from the user recursively
}

// Learning algorithm is initialized 
LearningAlgorithm algorithm = factory.createLearningAlgorithm(
Algorithm.RPNI, base, alphabetsize);


// Algorithm made to advance
BasicAutomaton automaton = (BasicAutomaton) algorithm.advance();

//produces the ".dot" code
make_OutputFile(automaton.toDot());
}
}
\end{lstlisting}




\section{Loggers}
\libalf's logger allows the learning algorithm to write information that may be relevant for application debugging. In fact, you may choose to make the learning algorithm write only those type of messages that is of interest to you. On the contrary, you may even choose to work without a logger by not associating any logger with learning algorithm during its initialization. 
\paragraph{}
\libalf essentially offers two types of loggers. 
\begin{enumerate}
 \item OutputStream Logger - The algorithm can be associated with any output stream.
 \item Buffered Logger - The algorithm can be associated with a buffer that can be used as a logger. In this case, the messages have to received (essentially to a string) and flushed.
\end{enumerate}
\paragraph{Types of Messages}
Every category or type of message has a priority associated with it. This is called a \texttt{loglevel}. The minimal loglevel defines the priority that is set. The following is the loglevel list. 
\begin{itemize}
 \item \textbf{LOGGER\_INTERNAL=0} ; (An internal method)
 \item \textbf{LOGGER\_ERROR = 1} ; All log messages that describe a non-recoverable error are marked with this.
 \item \textbf{LOGGER\_WARN = 2} ; Messages describing a state or command that is erroneous but may be ignored under most conditions.
 \item \textbf{LOGGER\_INFO = 3} ; Any information that does not describe an erroneous condition.
 \item \textbf{LOGGER\_DEBUG = 4} ; Messages that may help debugging of libalf.( Most likely removed before release version ).
 \item \textbf{LOGGER\_ALGORITHM = 5} ; (Do not use this as minimal loglevel)
\end{itemize}
For instance, setting up a \texttt{loglevel} of ``2'' will make the learning algorithm write warning and error messages (level 1 and 2) to the logger while messages labelled with lower priority (3 and 4) and discarded.

\paragraph{Other Components of \libalf}
\begin{enumerate}
 \item \textbf{Statistics} - Statistical data of the learning algorithm with respect to memory usage, query production (for online algorithms) and timing.
 \item \textbf{Filters} - Reduces the number of queries asked to a teacher by exploiting domain specific properties when associated with a knowledgebase.
 \item \textbf{Normalizers} - Recognizes equivalent words in a domain-specific sense to reduce the amount of knowledge that has to be stored.
\end{enumerate}

\section{Features of \libalf}
The library also offers other features as listed below.
\begin{enumerate}
 \item \textbf{GraphViz visualization} \\
       \libalf can create a GraphViz visualization of the knowledgebase and conjecture. The ``.dot`` file can be executed using the GraphViz tool. The latest release can be found in the GraphViz official web page (\url{www.graphviz.org}).
 \item \textbf{Serialization and Deserialization} \\
	At any point of time, the state of the knowledgebase, learning algorithms and other components can be serialized to a linear representation which you may store locally, share it in the internet or deserialize it in other machine for use.
 \item \textbf{Merging knowledgebases} \\
	\libalf also allows you to merge your knowledgebase with another one with certain constraints.
\end{enumerate}
Please refer to the main documentation to know more about \libalf and all other features.

